\chapter{結論}

\section{研究の成果の要約}

本論文では、急速に発展するAI技術に対する法規制の在り方について、諸外国の動向と比較しつつ、日本がとるべき道を検討してきた。
EUの厳格なリスク管理、中国の国家主導による統制、アメリカのイノベーション優先のアプローチ、そして韓国の柔軟性である。
それぞれの比較分析から明らかになったのは、世界各国がそれぞれの国益と価値観に基づき、「イノベーションの促進」と「リスクの制御」の両立を実現しようとしているという事実であった。

これに対して、日本において施行された「AI推進法」はイノベーションへの過度な配慮から、リスク管理における法的拘束力を欠いた内容となっている。
しかし、世界各国で行われているAIについての議論や、AIが孕む倫理的・社会的リスク(差別、偽情報、操作等)を考慮すれば、実効性のある規制を行わないことは、むしろ極めて危険な行為であると言わざるを得ない。

この課題を解決するために、本論文では4つの柱からなる新たな規制枠組みを提言した。

\begin{enumerate}
  \item リスクベースアプローチの導入 : 憲法に違反するレベルのAIを禁止し、高リスクAIを特定して管理する。
  \item 域外適用の明文化 : 法の適用範囲を国外事業者にも拡張し、法的な公平性を担保する。
  \item 透明性の義務化 : 説明可能性を重要視し、生成AIコンテンツへの表示等を義務付ける。
  \item 「実効性の担保」と「イノベーションの調和」 : 実効性のある「罰則」を設ける一方で、「規制サンドボックス」によりイノベーションの場を保証する。
\end{enumerate}

これら4つの施策は、決して過度な要求ではない。国民の権利と安全を守りつつ、国家の発展をするために不可欠な「ミニマム・スタンダード(最低基準)」である。

日本が「世界で最もAIを開発・活用しやすい国」を目指すことは否定しない。しかし、それは決して「国民の安全を軽視する国」であってはならない。
本提言による規律の導入は、日本が責任あるAI国家として国際社会でリーダーシップを発揮するための必須条件である。

\section{本研究の限界と今後の課題}
本研究では、各国の動向を踏まえた日本におけるAI規制のあり方を提言したが、残された課題も存在する。

第一に、規制導入に伴う経済的影響の検証である。本研究は法制度の設計に主眼を置いたため、提言した規制が企業活動に与える具体的な影響、特にコンプライアンス体制の構築にかかるコストや、それがイノベーションに与える萎縮効果についての定量的な分析には至っていない。
これらについては、経済学的な視点も含めた多角的な検討が必要である。

第二に、国際的な法執行の実効性の担保である。本稿では法の域外適用や国際協調の必要性を論じたが、法的拘束力を持たない国や事業者に対し、具体的にどのように規制を遵守させるかという実務上の課題は残る。
しかし、AIのリスクが国境を越える以上、国際的な協力体制の構築が不可欠であるという結論に変わりはない。

以上の課題は残るものの、本研究で示した「規制」と「イノベーション」の両立という方向性は、今後のAIガバナンス議論において重要な土台となると考える。

\section{結び}

AI技術の社会実装が進む中で、我々の社会の在り方は大きく変化をしている。AIは単なるツールを超え、人間の認知や意思決定に深く介入する環境そのものとなりつつある。
このような時代において、AIを社会発展のための強力なパートナーとし、真に豊かな社会を実現するためには、適切な規制による秩序ある発展が必要不可欠である。
適切な規制とは、イノベーションを縛る鎖ではなく、イノベーションを正しい方向へと導くための道しるべである。

日本は、「規制なき自由」という安易な道を選ぶべきではない。「適切なイノベーションを促進するためのハードロー」という土台の上で、AIと共存する社会形成を推進する道を選択すべきである。
本論文の提言が、そのための法制度設計の一助となれば幸いである。