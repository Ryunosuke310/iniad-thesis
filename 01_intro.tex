\chapter{はじめに}

\section{研究の背景}
2006年にディープラーニングが提唱されてから、想像をはるかに超える速度でAIは発展をしてきている。
また最近では、生成AIというテキスト、画像、音声などを新たに生成することのできるAIも利用する人口が増えている。
それに伴い、AIは医療や教育分野など、様々な領域での利用がされている。
今後も更なる発展と社会への浸透が予想されているAIについて、私たちはどう向き合っていくべきなのだろうか。


\section{本論文の目的}
本論文は、技術の発展と社会への浸透がさらに進むと考えられているAIに対して、各国がどのように規制をかけているのかについて比較し、
日本において、どのようなAI規制を行っていくべきなのか検討すること。そして、日本のAI規制法に対して提言を行うことを目的としている。

