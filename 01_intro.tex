\chapter{はじめに}

\section{研究の背景}
ディープラーニングが提唱された2006年以降、AIは想像をはるかに超える速度で発展を続けている。
特に生成AIは、テキスト、画像、音声などを高精度で生成できるようになり、その性能は驚くほど向上した。
このようなAIは、これまで人間が行ってきた作業を代替するだけでなく、それ以上の新たな価値を創造できると期待されている。
実際に、医療や教育といった多岐にわたる分野で活用が進んでおり、今後も利用は拡大していくと考えられる。

一方で、AIの利用には多様なリスクが伴う。例えば、ハルシネーションと呼ばれる誤った情報の生成や、
プロセスのブラックボックス化による透明性の欠如といった問題が指摘されている。
また、高精度なフェイクコンテンツが作成可能になったことで、悪意ある利用が社会に甚大な損害をもたらす可能性もある。
AIは社会に大きな恩恵をもたらす有用なツールであると同時に、利用方法を適切に規制できなければ有害なものへと転じうる。
したがって、その両面性を理解した上で、適切な制度設計を行うことが喫緊の課題となっている。


\section{研究の目的}
本研究の目的は、今後社会に一層浸透していくと予想されるAIに対する各国の制度設計を比較分析することにある。
具体的には、EU、アメリカ、韓国といった主要国のAI関連法案を横断的に調査し、その違いや共通点を明らかにする。
この比較分析から得られた知見を基に、日本のAI法案(AI法)について考察する。
そして、AIの利便性と安全性を両立させ、日本の社会特性に適したAI規制のあり方を提言することを最終的な目的とする。


