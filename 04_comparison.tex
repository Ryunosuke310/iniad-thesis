\chapter{各国のAI規制状況の比較}

前章では、AIに対する法規制の必要性を論じた。それを受け本章では、主要国・国際組織におけるAI規制の具体的な動向を比較分析する。

具体的には、包括的な法制化を進めるEU、分野別の規律を重視するアメリカ、そして韓国、中国、さらには国連における議論を取り上げ、そのアプローチの違いと共通点を明らかにする。

この比較検討を通じて、後の章で展開する日本のAI規制への提言に向けた、多角的な示唆を得ることを目的とする。

\section{諸外国との比較}

AIへの規制アプローチは、国や地域の事情を反映して多様な姿を見せている。本節ではその具体的な動向を概観するにあたり、まず世界で最も包括的かつ先行する法規制として知られるEUの「AI Act」を分析の起点とする。
これを基準とすることで、アメリカの分野別アプローチや、アジア各国の独自のアプローチとの違いがより明確になるからである。
%(※この章でどのようなものさしで比較するかについて述べる方法もあり)

\subsection{EU}

\subsection{アメリカ}

\subsection{中国}

\subsection{韓国}

\subsection{国連}

\section{各国のアプローチの違いと、メリットデメリットについて}
