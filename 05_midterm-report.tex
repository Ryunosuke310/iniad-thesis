\chapter{中間とりまとめの分析と比較考察}

\section{関係主体について}

\subsection{主な主体}
中間とりまとめでは、AIに関わる主体を「AI開発者」「AI提供者」「AI利用者」の3つに分類して整理している。
それぞれの定義は以下の通りである。

\begin{itemize}
  \item AI開発者 : AIモデルの学習やシステム基盤の構築など、AIそのものの開発を担う主体。
  \item AI提供者 : 開発されたAIを既存のシステム等に組み込み、サービスとして提供する主体。
  \item AI利用者 : 提供されたAIサービスを業務等で活用、または最終的に利用する主体。
\end{itemize}

\subsection{国外事業者の位置づけ}
上記の3主体に加え、中間とりまとめでは国外事業者の扱いも重要な論点とされた。
その背景には、
(1)国内で利用される生成AIの多くが国外で開発・提供されている実態、
(2)インターネットを介して誰もが容易に国外サービスにアクセスできる現状、そして
(3)規制を国内事業者に限定した場合に生じる競争上の不利益(いわゆる不公正競争)への懸念がある。

これらの点を踏まえ、規制の実効性と公平性を確保する観点から、
国外事業者も国内事業者と同様に本制度の対象とすべきであるとの結論が示された。


\section{ハードローとソフトローの適切な組み合わせ}
中間とりまとめでは、まずAIがもたらすリスクに対し、多くの既存法令が一定の役割を果たすことを確認している(図〇参照)。
その上で、技術の進展や新たなリスクに対応するための追加的な制度の必要性が論じられた。

特に重要な論点として、法令による規律(ハードロー)とガイドライン等による柔軟な対応(ソフトロー)を
いかに適切に組み合わせるかが挙げられている。
これは、AIのリスクに対して各所管省庁が個別の法令やガイドラインで対応してきた現状を踏まえたものである。

ハードローとソフトローには、それぞれ以下のような利点と課題がある。

\begin{itemize}
    \item ハードロー(法令) : 罰則等を伴うため高い実効性を確保できる一方、厳格な規制が技術革新を阻害する「イノベーションの阻害」という課題を持つ。
    \item ソフトロー(ガイドライン等 ): 技術の急速な変化に迅速かつ柔軟に対応できる利点があるが、遵守が事業者の自主性に委ねられるため実効性の確保が難しいという課題を抱える。
\end{itemize}

このトレードオフを認識した上で、とりまとめは基本的な方針として、まずは事業者の自主的な取り組みを尊重するソフトローを基本とし、
生命・身体の安全に関わるなど、自主的な対応だけでは不十分な限定的な領域に限り、ハードローによる規制を検討するという方向性を示した。


\section{リスクベース・アプローチの導入と評価軸}
AIは、その利用分野や形態によって生じるリスクが多種多様である。
そのため、全てのAIに一律の規制を課すことは、イノベーションを阻害し経済発展の妨げとなる懸念が指摘された。

この課題に対する解決策として、中間とりまとめでは「リスクベース・アプローチ」の導入が提言されている。
これは、AIの利用分野や文脈に応じてリスクの程度を評価し、そのリスクに見合った規律を課すという考え方である。

このアプローチは、前節で述べたハードローとソフトローの組み合わせを具体化するものであり、以下のように整理できる。

\begin{itemize}
    \item リスクが低い領域 : 事業者の自主性を尊重し、ソフトロー(ガイドライン等)による対応を基本とする。
    \item リスクが高い領域 : 個人の生命や身体の安全を脅かす、あるいは重大な人権侵害や犯罪につながる可能性があるAIについては、ハードロー(法令)による規制の導入を検討する。
\end{itemize}

このように、リスクの程度に応じて規律の強弱を柔軟に組み合わせることで、安全確保とイノベーション促進の両立を目指す方針が示された。
