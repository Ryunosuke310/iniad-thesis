\documentclass[bachelor]{INIAD}%卒論用 
\addtolength{\footskip}{8mm}
\bibliographystyle{jplain} 
%\usepackage[dviout]{graphicx}
\usepackage[dvipdfmx]{graphicx}
\usepackage{bm}
\usepackage{amsmath}
\usepackage{ascmac}

%\usepackage{geometry}
%\geometry{left=30mm,right=30mm,top=35mm,bottom=30mm}

%\documentclass[oneside]{suribt}% 本文が * ページ以下のときに (掲示に注意)
\title{AI規制法について}
%\titlewidth{}% タイトル幅 (指定するときは単位つきで)
\author{佐藤 隆之助}
\eauthor{Ryunosuke Sato}% Copyright 表示で使われる
\studentid{1F99999999}
\supervisor{阪本 泰男 先生}% 1つの引数をとる (役職まで含めて書く)
%\supervisor{指導教員名 役職 \and 指導教員名 役職}% 複数教員の場合,\and でつなげる
\handin{2021}{1}% 提出月. 2 つ (年, 月) 引数をとる
%\keywords{キーワード1, キーワード2} % 概要の下に表示される
\renewcommand{\baselinestretch}{1.25}
\setcounter{tocdepth}{2}

\begin{document}
\mojiparline{40}
\maketitle%%%%%%%%%%%%%%%%%%% タイトル %%%%

\frontmatter% ここから前文

%\etitle{Title in English}

%\begin{eabstract}%%%%%%%%%%%%% 概要 %%%%%%%%
% 300 words abstract in English should be written here. 
%\end{eabstract}

\begin{abstract}%%%%%%%%%%%%% 概要 %%%%%%%%
 ここには論文要旨を記述します。論文要旨の書き方については、指導教員の指導を受けること。
\end{abstract}

%%%%%%%%%%%%% 目次 %%%%%%%%
{\makeatletter
\let\ps@jpl@in\ps@empty
\makeatother
\pagestyle{empty}
\tableofcontents
\clearpage}

\mainmatter% ここから本文 %%% 本文 %%%%%%%%

\chapter{はじめに}

\section{研究の背景}
ディープラーニングが提唱された2006年以降、AIは想像をはるかに超える速度で発展を続けている。
特に生成AIは、テキスト、画像、音声などを高精度で生成できるようになり、その性能は驚くほど向上した。
このようなAIは、これまで人間が行ってきた作業を代替するだけでなく、それ以上の新たな価値を創造できると期待されている。
実際に、医療や教育といった多岐にわたる分野で活用が進んでおり、今後も利用は拡大していくと考えられる。

一方で、AIの利用には多様なリスクが伴う。例えば、ハルシネーションと呼ばれる誤った情報の生成や、
プロセスのブラックボックス化による透明性の欠如といった問題が指摘されている。
また、高精度なフェイクコンテンツが作成可能になったことで、悪意ある利用が社会に甚大な損害をもたらす可能性もある。
AIは社会に大きな恩恵をもたらす有用なツールであると同時に、利用方法を適切に規制できなければ有害なものへと転じうる。
したがって、その両面性を理解した上で、適切な制度設計を行うことが喫緊の課題となっている。

%ここで日本のAI戦略会議の流れを軽く説明した方がよいかも?

\section{研究の目的}
本稿の目的は、主要国(EU、アメリカ、韓国)のAIに関する制度設計を比較分析し、
それを通じて日本の社会特性に適したAI規制のあり方を提言することにある。
まず、AIになぜ規制が必要なのかという根源的な問いに対し、
アシロマ原則や生命倫理といったこれまでの議論を基に理論的考察を行う。
次に、この考察を土台として主要国のAI関連法案を横断的に調査し、その共通点と相違点を明らかにする。
最終的に、これらの国際比較と理論的考察から得られた知見を統合し、
AIの利便性と安全性を両立させる日本のAI法案の具体的な姿を提示する。

\section{先行研究}


     % はじめに
\chapter{関連研究}
   % 関連研究
\chapter{提案手法}

\section{象の卵の大きさ}
外形を計測し,それが絶対的な卵の形の枠であるアルキメデスの円筒座標表示形(式(\ref{Archimedes}))と一致するかどうか調べる。もし一致していなければ、卵でない可能性がある。
\subsection{サブセクションのテスト}
\subsubsection{サブサブセクションのテスト}
%%
%% 以下,番号つきディスプレイ数式モードの例
%%
\begin{equation}
r(z)=0.5\sqrt{1-(e^z-2)^2}
\label{Archimedes}
\end{equation}
%%
%% inline数式モードの例
%%
ここで,$r$は$z$軸からの距離,$z$は$xy$に直行する軸方向で,直交座標系は右手座標系であるとする。すなわち,$\hat{\bm{x}}$,$\hat{\bm{y}}$,$\hat{\bm{z}}$を,それぞれ$x$,$y$,$z$軸方向の単位ベクトルとしたとき,$\hat{\bm{z}} = \hat{\bm{x}} \times \hat{\bm{y}}$が成立している。
また,$r$の偏微分$\frac{\partial r}{\partial z}$は...
%%
%% 式番をまとめて表示する例
%%

ところで,カモメ(図\ref{kamome})が$x$羽,象が$y$匹としたときに,頭の数と,足の数の関係から,
\begin{eqnarray}
  \begin{cases}
    x + y = 2 & \\
    2x + 4y = 6 &
  \end{cases}
\end{eqnarray}

%%
%% 図の入れ方の例
%%
\begin{figure}[tb]
  \begin{center}
   \includegraphics[width=0.4\linewidth]{image/kamome.jpg}
  \end{center}
  \caption{カモメ}
  \label{kamome}
\end{figure}
  % 提案手法
\chapter{各国のAI規制状況の比較}

% 以降、実装や評価、結論などの章を適切に配置してください

\backmatter% ここから後付
\chapter{謝辞}
本研究の遂行にあたり、アンケートへの回答にご協力いただいた皆様に感謝いたします。
           % 謝辞

\bibliography{thesis.bib}  % 参考文献

\appendix% ここから付録 %%%%% 付録 %%%%%%%
\chapter{評価に利用したコード例}
      % 付録

\end{document}
