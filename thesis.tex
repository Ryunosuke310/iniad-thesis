\documentclass[bachelor]{INIAD}%卒論用 
\addtolength{\footskip}{8mm}
\bibliographystyle{jplain} 
%\usepackage[dviout]{graphicx}
\usepackage[dvipdfmx]{graphicx}
\usepackage{bm}
\usepackage{amsmath}
\usepackage{ascmac}

%\usepackage{geometry}
%\geometry{left=30mm,right=30mm,top=35mm,bottom=30mm}

%\documentclass[oneside]{suribt}% 本文が * ページ以下のときに (掲示に注意)
\title{AI規制法について}
%\titlewidth{}% タイトル幅 (指定するときは単位つきで)
\author{佐藤 隆之助}
\eauthor{Ryunosuke Sato}% Copyright 表示で使われる
\studentid{1F10220258}
\supervisor{阪本 泰男 先生}% 1つの引数をとる (役職まで含めて書く)
%\supervisor{指導教員名 役職 \and 指導教員名 役職}% 複数教員の場合,\and でつなげる
\handin{2025}{9}% 提出月. 2 つ (年, 月) 引数をとる
%\keywords{キーワード1, キーワード2} % 概要の下に表示される
\renewcommand{\baselinestretch}{1.25}
\setcounter{tocdepth}{2}

\begin{document}
\mojiparline{40}
\maketitle%%%%%%%%%%%%%%%%%%% タイトル %%%%

\frontmatter% ここから前文

%\etitle{Title in English}

%\begin{eabstract}%%%%%%%%%%%%% 概要 %%%%%%%%
% 300 words abstract in English should be written here. 
%\end{eabstract}

\begin{abstract}%%%%%%%%%%%%% 概要 %%%%%%%%
 ここには論文要旨を記述します。論文要旨の書き方については、指導教員の指導を受けること。
\end{abstract}

%%%%%%%%%%%%% 目次 %%%%%%%%
{\makeatletter
\let\ps@jpl@in\ps@empty
\makeatother
\pagestyle{empty}
\tableofcontents
\clearpage}

\mainmatter% ここから本文 %%% 本文 %%%%%%%%

\chapter{はじめに}

\section{研究の背景}
ディープラーニングが提唱された2006年以降、AIは想像をはるかに超える速度で発展を続けている。
特に生成AIは、テキスト、画像、音声などを高精度で生成できるようになり、その性能は驚くほど向上した。
このようなAIは、これまで人間が行ってきた作業を代替するだけでなく、それ以上の新たな価値を創造できると期待されている。
実際に、医療や教育といった多岐にわたる分野で活用が進んでおり、今後も利用は拡大していくと考えられる。

一方で、AIの利用には多様なリスクが伴う。例えば、ハルシネーションと呼ばれる誤った情報の生成や、
プロセスのブラックボックス化による透明性の欠如といった問題が指摘されている。
また、高精度なフェイクコンテンツが作成可能になったことで、悪意ある利用が社会に甚大な損害をもたらす可能性もある。
AIは社会に大きな恩恵をもたらす有用なツールであると同時に、利用方法を適切に規制できなければ有害なものへと転じうる。
したがって、その両面性を理解した上で、適切な制度設計を行うことが喫緊の課題となっている。

%ここで日本のAI戦略会議の流れを軽く説明した方がよいかも?

\section{研究の目的}
本稿の目的は、主要国(EU、アメリカ、韓国)のAIに関する制度設計を比較分析し、
それを通じて日本の社会特性に適したAI規制のあり方を提言することにある。
まず、AIになぜ規制が必要なのかという根源的な問いに対し、
アシロマ原則や生命倫理といったこれまでの議論を基に理論的考察を行う。
次に、この考察を土台として主要国のAI関連法案を横断的に調査し、その共通点と相違点を明らかにする。
最終的に、これらの国際比較と理論的考察から得られた知見を統合し、
AIの利便性と安全性を両立させる日本のAI法案の具体的な姿を提示する。

\section{先行研究}


     
\chapter{AIの概要とリスク}
象の卵生については、すでに多くの研究がなされている。...

\section{AIの定義と種類}


\section{AIがもたらすリスクについて}   
\chapter{日本のAI規制をめぐるこれまでの議論と経緯}

\section{議論の始動と検討の枠組み}

\section{関係主体について}

\subsection{主な主体}
中間とりまとめでは、AIに関わる主体を「AI開発者」「AI提供者」「AI利用者」の3つに分類して整理している。
それぞれの定義は以下の通りである。

\begin{itemize}
  \item AI開発者:AIモデルの学習やシステム基盤の構築など、AIそのものの開発を担う主体。
  \item AI提供者:開発されたAIを既存のシステム等に組み込み、サービスとして提供する主体。
  \item AI利用者:提供されたAIサービスを業務等で活用、または最終的に利用する主体。
\end{itemize}

\subsection{国外事業者の位置づけ}
上記の3主体に加え、中間とりまとめでは国外事業者の扱いも重要な論点とされた。
その背景には、
①国内で利用される生成AIの多くが国外で開発・提供されている実態、
②インターネットを介して誰もが容易に国外サービスにアクセスできる現状、そして
③規制を国内事業者に限定した場合に生じる競争上の不利益(いわゆる不公正競争)への懸念がある。

これらの点を踏まえ、規制の実効性と公平性を確保する観点から、
国外事業者も国内事業者と同様に本制度の対象とすべきであるとの結論が示された。


\section{ハードローとソフトローの適切な組み合わせ}
中間とりまとめでは、まずAIがもたらすリスクに対し、多くの既存法令が一定の役割を果たすことを確認している(図〇参照)。
その上で、技術の進展や新たなリスクに対応するための追加的な制度の必要性が論じられた。

特に重要な論点として、法令による規律(ハードロー)とガイドライン等による柔軟な対応(ソフトロー)を
いかに適切に組み合わせるかが挙げられている。
これは、AIのリスクに対して各所管省庁が個別の法令やガイドラインで対応してきた現状を踏まえたものである。

ハードローとソフトローには、それぞれ以下のような利点と課題がある。

\begin{itemize}
    \item ハードロー(法令):罰則等を伴うため高い実効性を確保できる一方、厳格な規制が技術革新を阻害する「イノベーションの阻害」という課題を持つ。
    \item ソフトロー(ガイドライン等):技術の急速な変化に迅速かつ柔軟に対応できる利点があるが、遵守が事業者の自主性に委ねられるため実効性の確保が難しいという課題を抱える。
\end{itemize}

このトレードオフを認識した上で、とりまとめは基本的な方針として、まずは事業者の自主的な取り組みを尊重するソフトローを基本とし、
生命・身体の安全に関わるなど、自主的な対応だけでは不十分な限定的な領域に限り、ハードローによる規制を検討するという方向性を示した。


\section{リスクベース・アプローチの導入と評価軸}
AIは、その利用分野や形態によって生じるリスクが多種多様である。
そのため、全てのAIに一律の規制を課すことは、イノベーションを阻害し経済発展の妨げとなる懸念が指摘された。

この課題に対する解決策として、中間とりまとめでは「リスクベース・アプローチ」の導入が提言されている。
これは、AIの利用分野や文脈に応じてリスクの程度を評価し、そのリスクに見合った規律を課すという考え方である。

このアプローチは、前節で述べたハードローとソフトローの組み合わせを具体化するものであり、以下のように整理できる。

\begin{itemize}
    \item リスクが低い領域:事業者の自主性を尊重し、ソフトロー(ガイドライン等)による対応を基本とする。
    \item リスクが高い領域:個人の生命や身体の安全を脅かす、あるいは重大な人権侵害や犯罪につながる可能性があるAIについては、ハードロー(法令)による規制の導入を検討する。
\end{itemize}

このように、リスクの程度に応じて規律の強弱を柔軟に組み合わせることで、安全確保とイノベーション促進の両立を目指す方針が示された。

\section{最終報告 (AI法案について)の意見}
\chapter{各国のAI規制状況の比較}

前章では、AIに対する法規制の必要性を論じた。それを受け本章では、主要国・国際組織におけるAI規制の具体的な動向を比較分析する。

具体的には、包括的な法制化を進めるEU、分野別の規律を重視するアメリカ、そして韓国、中国、さらには国連における議論を取り上げ、そのアプローチの違いと共通点を明らかにする。

この比較検討を通じて、後の章で展開する日本のAI規制への提言に向けた、多角的な示唆を得ることを目的とする。

\section{諸外国との比較}

AIへの規制アプローチは、国や地域の事情を反映して多様な姿を見せている。本節ではその具体的な動向を概観するにあたり、まず世界で最も包括的かつ先行する法規制として知られるEUの「AI Act」を分析の起点とする。
これを基準とすることで、アメリカの分野別アプローチや、アジア各国の独自のアプローチとの違いがより明確になるからである。
%(※この章でどのようなものさしで比較するかについて述べる方法もあり)

\subsection{EU}

\subsection{アメリカ}

\subsection{中国}

\subsection{韓国}

\subsection{国連}

\section{各国のアプローチの違いと、メリットデメリットについて}

\chapter{AI規制法の必要性について}

\section{AI原則について}

\subsection{アシロマ原則}

\subsection{AIの研究開発の原則}

\subsection{G20AI原則}

\section{生命倫理とAI倫理}

\subsection{フロリディの生命倫理}

\subsection{生命倫理とAI倫理の接点}

\section{ソフトローとハードローの選択}

\subsection{ソフトロー}

\subsection{ハードロー}

\subsection{日本に合わせたアプローチ}
\chapter{日本のAI法案に対する提言}

\section{「最終報告(AI法案)」の概要と評価}

\section{規制項目の提案}

\section{罰則規定の導入}
\chapter{結論}

\section{日本のAI規制の在り方についての提言}

\section{技術革新の促進と安全・信頼性確保のバランスの重要性}

% 以降、実装や評価、結論などの章を適切に配置してください

\backmatter% ここから後付
\chapter{謝辞}
本研究の遂行にあたり、アンケートへの回答にご協力いただいた皆様に感謝いたします。
           % 謝辞

\bibliography{thesis.bib}  % 参考文献

\appendix% ここから付録 %%%%% 付録 %%%%%%%
%\chapter{評価に利用したコード例}
 %表示しないようにしてます      % 付録

\end{document}
