\chapter{その他の \LaTeX の機能についての紹介}
この章では、本編に含まれない \LaTeX の機能について、その使い方の一部をサンプルとして紹介する。

\section{コード例}
コードを図表に含める場合の例を図\ref{eval_code}に示す。

ここではそのまま利用できる \verb|verbatim| 環境を利用しているが、実際は \verb|jlisting| 等のパッケージを利用したほうがよい
\footnote{適切に設定すれば、行番号やきれいな枠をつけることができる}。

\begin{figure}[tb]
  \begin{screen}
    \begin{verbatim}int main(int argc, char* argv)
{
    printf("hello\n");
    return 0;
}\end{verbatim}
  \end{screen}
  \caption{評価に使用したコード例}
  \label{eval_code}
\end{figure}

\section{様々な箇条書き}
\LaTeX では様々な箇条書きを利用できる。

HTMLの \verb|ul| 要素に相当する順序なしの箇条書きには、\verb|itemize|環境を用いる。

\begin{itemize}
  \item りんご
  \item みかん
  \item バナナ
\end{itemize}

HTMLの \verb|ol| 要素に相当する順序ありの箇条書きには、\verb|enumerate|環境を用いる。

\begin{enumerate}
  \item 起
  \item 承
  \item 転
  \item 結
\end{enumerate}

HTMLの \verb|dl| 要素に相当する説明リストには、\verb|description|環境を用いる。

\begin{description}
  \item[麻婆丼] 500円
  \item[香港風カレー] 500円
  \item[マンゴープリン] 300円
\end{description}

\section{太字やイタリック}
フォントを指定する場合には \verb|textbf| や \verb|textit| 等の命令を用いる。

たとえば、\textbf{太字}にしたり、\textit{斜体}を指定することができる\footnote{ただし、日本語フォントの場合は指定通りの太字や斜体にならない場合もある}。